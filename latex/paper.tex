\documentclass[a4paper,10pt]{article}

\usepackage{subfig}
\usepackage{float}
\usepackage{graphicx}
\usepackage{verbatim}
\usepackage{amsthm}
\usepackage{amsmath}
\usepackage{geometry}

\newtheorem{defn}{Definition}
\newtheorem{prop}{Proposition}
\newtheorem{theorem}{Theorem}

%opening
\title{Degree-dependent Rewiring Model}
\author{Sam Magura, Vitchyr Pong}

\begin{document}

\maketitle

\section{Model Definition}

The process operates on a graph $G = (V, E)$. Each iteration:

\begin{enumerate}
 \item Randomly select an edge from $E$. 
 \item Select a random orientation $(x, y)$ for the edge.
 \item \label{item:z} Randomly select a vertex $z$ from $V \setminus \{x, y\}$.
 \item \label{item:pr} Create an edge $\{x, z\}$ with probability
\begin{equation}
 1 - e^{-c\,(\theta \: d(z) / \overline{d}\; + (1 - \theta))}
\end{equation}
where $d(z)$ gives the degree of $z$, $\overline{d}$ is the mean degree of the graph, $\theta$ is an adjustable parameter, and $c$ is some constant.
 \item If an edge was created in the previous step, remove the edge $\{x, y\}$.
\end{enumerate}

\section{Equilibrium Distribution}
Consider states $G$ and $H$ such that $\{x, y\}$ is in $G$ but not $H$, and $\{x, z\}$ is in $H$ but not $G$. 

\paragraph{Defintions} 
\begin{itemize}
 \item Let $n$ be the number of vertices let $m$ be the number of edges. Niether $n$ nor $m$ can change during an iteration, since for every new edge that is created, another is removed.
 \item  Let the degree of $x$, which is the same in both states, be given by $i$. Let the degree of $y$ in $G$ be $j$ and the degree of $z$ in $H$ by $k$. Thus the degree of $y$ in $H$ is $j - 1$, and the degree of $z$ in $G$ is $k - 1$.
 \item Let the probability of rewiring to a vertex of degree $i$ in step \ref{item:pr} of the model be given by $f(i)$. Thus,
\begin{equation}
 f(i) = 1 - e^{-c\,(\theta \: i / \overline{d}\; + (1 - \theta))}.
\end{equation}
 \item Let
 \begin{equation}
  F(i) = \left(\frac{1}{f(i)}\right)^i.
 \end{equation}

\end{itemize}

\paragraph{Transition probability} For a transition from $G$ to $H$, the following must occur.
\begin{enumerate}
 \item The edge $\{x, y\}$ is selected. This occurs with probability $\frac{1}{m}$.
 \item The orientation $(x, y)$ is chosen. This occurs with probability $\frac{1}{2}$.
 \item Vertex $z$ is selected in step \ref{item:z}. Since this vertex is chosen randomly from $V \setminus \{x, y\}$, the probability that $z$ is selected is $\frac{1}{m - 2}$.
 \item The rewiring is accepted in step \ref{item:pr}. This occurs with probability $f(k - 1)$.
\end{enumerate}

\noindent Therefore the transition probability is 

\begin{equation}
 P(G, H) = \frac{1}{2m} \: \frac{1}{m - 2} \: f(k - 1).
\end{equation}

\noindent By considering this argument for a transition from $H$ to $G$, we find that

\begin{equation}
 P(H, G) = \frac{1}{2m} \: \frac{1}{m - 2} \: f(j - 1).
\end{equation}

\paragraph{Detailed balance}
\begin{theorem}
 The detailed balance condition is satisfied by

\begin{equation}
 \pi(G) = \prod_{x = 1}^{n} F(d(x))
\end{equation}
where $d(x)$ gives the degree of $x$ in $G$. Therefore, $\pi$ is the equilibrium distribution.

\end{theorem}

\end{document}
