\documentclass[11pt,letterpaper]{article}

%%%%%%%%%%%%%%%%%%%%%%%%%%%%%%%%%%%%%%%%%%%%%%%%%%%%%%%%%%%%%%%%%%%%%%%%%
    \usepackage{amsmath}
\usepackage{amssymb}
\usepackage{amsthm}
\usepackage{graphicx}
\usepackage{bbm}
\usepackage{subfigure}

\pagestyle{plain}                                                     

%%%%%%%%%% MARGINS %%%%%%%   
\setlength{\textwidth}{6.5in}     %%
\setlength{\oddsidemargin}{0in}   %%
\setlength{\evensidemargin}{0in}  %%
\setlength{\textheight}{8.5in}    %%  
\setlength{\topmargin}{0in}       %%  
\setlength{\headheight}{0in}      %%  
\setlength{\headsep}{0in}         %%                           
\setlength{\footskip}{.5in}       %%                                   
%%%%%%%%%%%%%%%%%%%%%%%%%%%%%%%%%%%%                                  

\newtheorem{thm}{Theorem}
\newtheorem{con}{Conjecture}
\newtheorem{prop}[thm]{Proposition}
\newtheorem{lem}[thm]{Lemma}
\newtheorem{defn}[thm]{Definition}
\newtheorem*{theorem}{Theorem}
\newtheorem*{lemma}{Lemma}

    %For norms use \norm{a} this writes ||a||, for absolute value use \abs
     \newcommand{\norm}[1]{\left|\left|#1\right|\right|}
     \newcommand{\abs}[1]{\left|#1\right|}

     \newcommand{\prob}[1]{\mathbb{P}\left(#1\right)}
     \newcommand{\eps}[0]{\epsilon}
     \newcommand{\cluster}[0]{\mathcal{C}}
     \newcommand{\E}[0]{\mathbb{E}}
     \newcommand{\tab}[0]{\hspace{1.5em}}
     \newcommand{\I}[1]{\mathbbm{1}_{#1}}
     \newcommand{\floor}[1]{\left\lfloor#1\right \rfloor}
     \newcommand{\ceil}[1]{\left\lceil#1\right\rceil}
     \newcommand{\vect}[1]{{\bf #1}}
     \newcommand{\D}{\mathcal{D}}
            
\renewcommand{\refname}{\hfil References Cited\hfil}                   
\bibliographystyle{abbrv}                                              
%%%%%%%%%%%%%%%%%%%%%%%%%%%%%%%%%%%%%%%%%%%%%%%%%%

\begin{document}
\begin{center}
{\LARGE Long Range Percolation} \\
\today
\end{center}

Papers on long range percolation, in which an edge is drawn independently between vertices $i$ and $j$ with probability $p_{ij} = f(d(i,j))$, where $f: \mathbb{Z}_+ \to [0,1)$ is such that $f(d)d^s \to \beta>0$ as $d \to \infty$.

\begin{enumerate}
\item {\bf Schulman, 1983~\cite{Schulman:1983}}: He studies LRP on $\mathbb{Z}$ where $f(d) = \beta/d^s$ for $\beta \in (0,1)$.
\begin{enumerate}
\item For $1<s\leq 2$, if $\beta < 1/2\zeta(s)$ there is no infinite component.
\item For $s>2$ there is never an infinite component.
\end{enumerate}

\item {\bf Newman and Schulman, 1986~\cite{NS:1986}}: They study LRP on $\mathbb{Z}$ where $f(1)<1$.
\begin{enumerate}
\item If $\displaystyle \liminf_{d\to\infty} d^s f(d) >0$ for some $s<2$ then an infinite component exists if $f(1)$ is sufficiently close to $1$.
\item If $\displaystyle \liminf_{d\to\infty} d^2 f(d) >1$ then an infinite component exists if $f(1)$ is sufficiently close to $1$.
\end{enumerate}


\item {\bf Benjamini and Berger, 2001~\cite{BB:2001}}:  They study the diameter $\D(N)$ of LRP on the finite cycle $\mathbb{Z}/N\mathbb{Z}$, where $f(1) = 1$ (so nearest neighbors are connected) and $f(d) = 1 -\exp(-\beta d^{-s})$ otherwise.
\begin{enumerate}
\item If $s>2$ then there is $C$ such that
$$\prob{\D(N)\geq CN} \to 1 $$
and $\D(N)/N \to \eta \in (0,1]$ in distribution.

\item If $s<1$ then there is a constant $C$ such that
$$ \prob{\D(N)\leq C} \to 1. $$

\item If $1<s<2$ then there are constants $\delta$ and $C$ such that
$$ \prob{C \log N \leq \D(N) \leq (\log N)^\delta} \to 1.$$
(Another proof of the upper bound is in~\cite{CGS:2002})

\item The authors also prove an upper bound for the Cheeger constant of $N^{-a}$ ($a<s-1$) when $1<s<2$, and mention a result in~\cite{Berger:2002} that bounds the Cheeger constant for $s=2$ by $C\log N / N$.
\end{enumerate}

%%%%%%%%%%%%
\item {\bf Coppersmith, Gamarnik and Sviridenko, 2002~\cite{CGS:2002}}: They study the diameter $\D(N)$ of LRP on the $d$-dimensional finite grid $[N]^d$, where nearest neighbors are connected ($f(1)=1$).

\begin{enumerate}
\item For $s>2d$ and $\delta < \frac{s-2d}{s-d-1}$
$$\prob{\D(N) \geq N^\delta} \to 1.$$

\item For $s = 2d$ there exist $0<\eta_1<\eta_2<1$ such that
$$ \prob{\D(N) \leq N^{\eta_2}} \to 1,$$
and when $d=1, s=2, \beta<1$
$$ \prob{\D(N) \geq N^{\eta_1}}\to 1.$$

\item For $d<s<2d$ they extend the results of Benjamini and Berger to $d>1$, though their approach is different.  Their results are of course trumped by those of Biskup~\cite{Biskup:2004}.

\item For $s=d$
$$ \prob{\frac{C_1 \log N}{\log \log N} \leq \D(N) \leq \frac{C_2 \log N}{\log \log N}} \to 1$$

\end{enumerate}

%%%%%%%%
\item {\bf Biskup, 2004 and 2011~\cite{Biskup:2004, Biskup:2011}}: He studies LRP on $\mathbb{Z}^d$ and $[N]^d$.  On $\mathbb{Z}^d$ he studies the scaling of $D(x,y)$, the graph distance between $x$ and $y$, while on $[N]^d$ he studies the diameter $\D(N)$.  For $d<s<2d$, if $\Delta = \frac{\log 2}{\log(2d/s)}$ then
\begin{align*}
D(x,y) &= (\log d(x,y))^{\Delta + o(1)}  &\text{as } d(x,y) \to \infty,\\
\D(N) &= (\log N)^{\Delta + o(1)} &\text{as } N \to \infty,
\end{align*}
where $o(1)$ is a term that goes to zero in probability.


\item {\bf Apparently Still Open}: 
\begin{enumerate}
\item For $s=2d$ is it true that $\D(N) \sim N^{\eta}$?
\item What is the precise percolation threshold in terms of $\beta$?  Size the giant component in finite graphs?
\item What happens outside the range $s\in (d,2d)$ if the assumption of nearest neighbor connections is relaxed?
\end{enumerate}

\end{enumerate}

\bibstyle{siam}
\bibliographystyle{siam}
\bibliography{references}

\end{document}  