\documentclass[a4paper,10pt]{article}

\usepackage{subfig}
\usepackage{float}
\usepackage{graphicx}

%opening
\title{A Social Network Model}
\author{Sam Magura, Vitchyr Pong}

\begin{document}

\maketitle

\section{Introduction}

In this social network model, we choose a node and break one of its incident edges. The model is speical in that the node is less likely to break a connection with a high-degree neighbor. This behavior reflects real social behavior; relationships with popular, well-connected people are potentially more valuble than relationships with less well-connected people. The model takes a parameter $I$ --- a measure of \emph{intolerance} toward unpopular people --- that determines the extent to which the degrees of adjacent nodes affect the breaking probabilities. We predicted that changing $I$ make the model behave differently, but our simulations showed no relationship between $I$ and the model's macroscopic behavior. 

\section{Model Definition}

The process operates on a graph $G = (V, E)$. An iteration consists of the following steps. 

\begin{enumerate}
 \item Chose a vertex $v_0$ randomly from the set of all vertices with degree 1 or greater. 
 \item Exactly one edge incident to $v_0$ will be broken. For an edge $e^*$ that is incident to $v_0$, let the chance that $e^*$ is broken be proportional to

 \begin{equation}
\label{eqn:pr-function}
  I \frac{1}{d^*} + (1 - I)\frac{1}{deg(v_0)}
 \end{equation}

where $d^*$ is the degree of the \emph{other} vertex that $e^*$ is incident on. Let this equation be known as the probability function. 

 \item Select a vertex $v'$ from $V \setminus \{v_0\}$. If there already exists an edge $(v_0, v')$, select a new $v'$ from $V \setminus \{v_0\}$. Continue to select vertices until a $v'$ is selected that is not adjacent to $v_0$.

 \item Add an edge $(v_0, v')$ to the graph. 

\end{enumerate}

\section{Results}

We implemented the model in the Python programming language. In our implementation, the initial graph was $G(n, M)$ Erdos-Renyi random graph. We wrote an additional script to run the model many times with different values $I$ to see if the behavior of the model depended on $I$ is a significant way. The script performed 1000 iterations for each trial. It then gathered several statistics on the resultant graph: number of nodes in largest component, diameter of largest component, and number of components. We predicted that the size and diameter of the largest component would increase with $I$, while the number of components would decrease with $I$. 

\begin{figure}[H]
\begin{center}
\subfloat{\includegraphics[height=4cm]{images/gc_size.png}}
\subfloat{\includegraphics[height=4cm]{images/gc_diameter.png}}
\caption{Size and diameter of the largest component, for 40 nodes, 20 edges, and 1000 iterations per trial.}
\end{center}
\end{figure} 

\begin{figure}[H]
\begin{center}
\subfloat{\includegraphics[height=4cm]{images/n_components.png}}
\subfloat{\includegraphics[height=4cm]{images/n_components_mean.png}}
\caption{Number of components and 20-trial means for number of components --- using 40 nodes, 20 edges, and 1000 iterations per trial.}
\end{center}
\end{figure} 

We did not notice a substantial correlation between $I$ and the size of largest component, the diameter of the largest component, or the number of components. There appears to be a very slight negative correlation between $I$ and the 20-trial number-of-component means.

%Vitchyr's ``sine wave''!

\subsection{Model Variations}
In attempt to get more interesting results, we tried several changes to the model. None of the changes produced correlations between $I$ and any of the three quantities we recorded.

\begin{itemize}

 \item We tried to change the probability function to increase the effect of $I$ on the breaking probabilities. For example,
 \begin{equation}
  I^2 \frac{1}{d^*} + (1 - I)^2\frac{1}{deg(v_0)}
 \end{equation}
and
 \begin{equation}
  sin(I) \frac{1}{d^*} + cos(I)\frac{1}{deg(v_0)}.
 \end{equation}

 Neither of these probability functions produced interesting results.

 \item Instead of using a probability function, we modified the model so that $v_0$ would always break its connection with its neighbor of least degree --- and if there were multiple neighbors of least degree, chose which one to break with randomly. 

\end{itemize}


\end{document}
