\documentclass[a4paper,10pt]{article}

%opening
\title{}
\author{}

\begin{document}

\section{Model Definition}

The following is done in a single iteration. The process operates on a graph $G = (V, E)$.

\begin{enumerate}
 \item Chose a vertex $v_0$ randomly from the set of all vertices. 
 \item Do nothing if $deg(v_0) = 0$. 
 \item For some edge $e^*$ that is incident to $v_0$, break $e^*$ with probability

 \begin{equation}
  I \frac{c}{d^*} + (1 - I)\frac{1}{deg(v_0)}
 \end{equation}

where $d^*$ is the degree of the \emph{other} vertex that $e^*$ is incident to. 

 \item Add a new edge $\{v_0, v'\}$ with $v'$ chosen randomly from $V \setminus \{v_0\}$.

\end{enumerate}

In the probability function, $c$ is some normalizing coeffecient such that $\frac{c}{d^*} \leq 1$. This is sufficient to normalize the entire probability function. The variable $I$ represents \emph{intolerance}. If $I = 0$, each edge has the same chance of being broken. If $I = 1$, the chance that $e^*$ is broken is $\frac{c}{d^*}$.

\end{document}
